	\chapter*{Introdução}
\addcontentsline{toc}{chapter}{INTRODUÇÃO}

Problemas Inversos é uma classe que reúne diversos problemas advindos de modelagens físicas, dos mais variados cenários e aplicações, nos quais busca-se determinar ou reconstruir um objeto conhecendo seu efeito através de algum processo conhecido. Matematicamente, como descreve \citeonline{kirsch}, tais problemas podem ser caracterizados da seguinte forma: dados dois espaços $X,Y$, um operador $K: X \to Y$ e determinado $y\in Y$ conhecido, queremos determinar $x\in X$ tal que $K(x) = y$. Isto é, busca-se uma solução para tal equação. Em especial, um dos problemas inversos de grande interesse e que será foco deste trabalho é a chamada Tomografia por Impedância Elétrica.

Tomografia por Impedância Elétrica (TIE) é um problema descrito da seguinte forma: em um determinado corpo são aplicadas uma série de correntes elétricas em sua superfície com amperagens conhecidas. A partir delas, medindo os potenciais resultantes, tenta-se reconstruir uma imagem de seu interior, em especial de sua impedância elétrica \cite{somersalo}. Tal problema possui uma grande variedade de aplicações e vantagens sobre outros métodos de tomografia, o que o torna um objeto de estudo de grande interesse. 

\citeonline{cheney} relatam, por exemplo, que uma das principais aplicações é na área médica, tendo vários usos possíveis para diagnóstico por imagem, podendo ser vantajosa devido a não necessidade de exposição a materiais e fenômenos radioativos, como a usual Tomografia por Raios-x. Outras aplicações citadas são a determinação de depósitos minerais no interior da terra, rastreamento da propagação de contaminantes na terra, avaliação não-destrutiva de componentes de máquina e controle de processos industriais \cite{cheney}. Além dessas, há também a motivação para o uso de tal tomografia neste trabalho: a imagem de escoamento multifásico.

Este trabalho surge de um projeto de pesquisa desenvolvido em parceria entre o Departamento de Matemática da Universidade Federal de Santa Catarina e o Instituto Fedeeral de Santa Catarina, com apoio do CNPQ, denominado “Impedance Tomography for monitoring multiphase flows” \cite{margotti-eit}. Neste projeto, o objetivo principal é desenvolver um sistema de Tomografia por Impedância Elétrica mirando o monitoramento de fluidos multifásicos, analisando a composição dos fluidos passando por uma tubulação ou duto, aplicação especialmente interessante na exploração de petróleo e similares. Tendo o objetivo de obter reconstruções precisas e de forma otimizada, surge a necessidade de estudar e implementar métodos eficientes para resolução do problema inverso em questão.

Como \citeonline{kirsch} descreve, a TIE pode ser caracterizada como um problema inverso, sendo este determinar o conteúdo do interior de um corpo conhecendo os potenciais resultantes após esse ser atravessado por certas correntes. Entretanto, nas modelagens mais comuns, é possível identificar a TIE como um problema inverso mal-posto, pois não apresenta certas condições que garantem uma estabilidade na aproximação de soluções \cite{kirsch}. Com essa instabilidade, torna-se necessário o estudo de técnicas mais rebuscadas para determinar aproximações mais precisas para o problema, em especial dos chamados Métodos de Regularização.

Métodos de Regularização, também chamadas Estratégias de Regularização, são famílias de operadores, conjuntos específicos de funções dependentes de certos parâmetros, que buscam obter soluções de forma suficientemente aproximada para um problema inverso, evitando sua instabilidade \cite{kirsch}. A aplicabilidade e eficiência de cada Método de Regularização depende das propriedades do problema inverso em questão, tornando-se preciso tanto estudar as características de cada método e problema inverso, bem como avaliar através de experimentos como se comportam em conjunto.

Considerando a motivação e as problemáticas apresentadas, o seguinte trabalho tem por objetivo estudar e comparar diferentes Métodos de Regularização aplicados ao problema da Tomografia por Impedância Elétrica, explorando os aspectos teóricos desses métodos e avaliando suas aplicações no problema. O trabalho se estruturará inicialmente abordando a Teoria geral de Regularização e Problemas Inversos, em seguida desenvolvendo estudos sobre os Métodos de Regularização que serão trabalhados, descrevendo as principais modelagens da TIE e suas propriedades, e, por fim, realizando experimentos numéricos com uso de dados simulados e reais.
	
