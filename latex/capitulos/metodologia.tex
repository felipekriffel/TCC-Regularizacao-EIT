\chapter{Metodologia}

Para realização do seguinte trabalho, serão realizados encontros semanais em conjunto com o orientador para estudos e discussões a respeito dos tópicos abordados, para além da consulta na bibliografia relacionada. 

Os experimentos numéricos serão realizados com uso de implementações computacionais disponíveis da Tomografia da Impedância Elétrica, desenvolvidas em trabalhos anteriores, em particular do sistema desenvolvido por \citeonline{hafemann}, implementado na linguagem de programação Python com apoio de bibliotecas para solução computacional de equações diferenciais. A avaliação será dada por medidas de desempenho de cada método de regularização, comparando entre os conjuntos de dados utilizados.

Os dados para experimentação serão de dois tipos: dados simulados e dados reais. Os dados simulados serão gerados artificialmente com uso das implementações computacionais da TIE, em especial na ferramenta desenvolvida e descrita por \citeonline{hafemann}, onde há funções para gerar os dados de potenciais resultantes através de determinadas correntes e condutividade especificadas. Os dados reais são providos de um sistema, coletados em experimentos conduzidos pelo projeto "Impedance Tomography for monitoring multiphase flow", cujo processo de coleta é descrito em \cite{margotti-eit}.

Dada a metodologia estipulada, na seção a seguir é descrito o cronograma para elaboração dos procedimentos citados.