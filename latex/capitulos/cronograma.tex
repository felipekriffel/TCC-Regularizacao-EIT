\chapter{Cronograma}

Nesta seção descreve-se o cronograma das atividades desenvolvidas para realização deste trabalho. Constam aquelas já executadas previamente ao longo dos anos de 2022 e 2023, bem como aquelas planejadas para serem executadas em momentos posteriores.

Este trabalho surge de um projeto de Iniciação Científica, vigente desde abril de 2022 e com encerramento previsto para dezembro de 2023, a qual se insere no contexto do projeto "“Impedance Tomography for monitoring multiphase flow" \cite{margotti-eit}. Nesse trabalho de pesquisa, ao longo de todo o período de execução, foram desenvolvidos estudos prévios em torno de muitos dos tópicos necessários à realização deste trabalho. 

Primeiramente, foram realizados encontros semanais ao longo do primeiro semestre de 2022, nos quais eram feitos seminários e apresentações pelo orientador e outros estudantes do projeto. Neles, foram discutidos e estudados diferentes aspectos das modelagens da TIE, as formulações variacionais equivalentes, sua implementação computacional via Método de Elementos Finitos, e os diferentes métodos numéricos para resolução dos problemas direto e inverso associados.  

Entre os meses de junho e outubro de 2022, ocorreram também reuniões semanais para estudos em torno de fundamentos da Análise Funcional. Nelas, que ocorreram em conjunto com o orientador e outro colega de graduação na época, eram feitas apresentações pelos orientandos de seções do livro "Introductory functional analysis with applications" \cite{kreyszig}, discutindo e desenvolvendo os conceitos e resultados trazidos no livro. Em particular, foram abordados tópicos sobre: Sequências de Cauchy em Espaços Métricos e Espaços Completos; Espaços Vetoriais Normados e de Banach; Operadores Lineares e Limitados; Espaços de Produto Interno e de Hilbert; Representações de Funcionais e Operadores Adjuntos em Espaços de Hilbert; Espaços Reflexivos; e Convergência Fraca.

Além dos encontros citados, foram feitos discussões e trabalhos paralelamente em torno da implementação computacional para o problema da TIE ao longo de todo o ano de 2022, ocorrendo também até o presente momento. Essa implementação, desenvolvida em trabalhos anteriores por \citeonline{hafemann} para o projeto de pesquisa citado, traz um programa utilizando a linguagem de programação \textit{Python} que executa a resolução dos problemas direto e inverso associados à TIE. Nessas ocasiões, estudou-se o uso de tal implementação para realização de diferentes testes numéricos em torno do problema, bem como foram feitos trabalhos para melhorias e aplicações dessa implementação existente. 

Já a Tabela \ref{tab:cronograma} descreve o cronograma de atividades para elaboração deste trabalho ao longo do ano de 2023. Nela está presente o planejamento para os meses posteriores, e também relata as etapas já desenvolvidas ao longo deste ano.

\begin{table}[htb]
\caption{Cronograma para o TCC referente ao ano de 2023}
\label{tab:cronograma} 
\begin{center} 
\def\arraystretch{3}%
\begin{tabular}{ | m{0.2 \textwidth} | m{0.7 \textwidth} | } 

\hline

Mês & Tópicos/Atividades \\

\hline 
  % Abril a Dezembro de 2022 &
  %   \begin{itemize}
  %       \item Encontros para estudo e discussão de fundamentos de Análise Funcional;
  %       \item Encontros para discussões sobre a modelagem e implementações da TIE;
  %       \item Estudos e experimentos com as implementações disponíveis da TIE;
  %   \end{itemize}
  %   \\
 
 \hline
 Março & 
    Estudos e encontros para discussão da Pseudo-Inversa de Moore-Penrose.\\ 

 \hline
 Abril &
    Encontros para discussão da Pseudo-Inversa de Moore-Penrose; 
    Encontros para estudos e discussão acerca de Métodos de Regularização;\\ 

 \hline
 Maio & 
    Encontros para estudos e discussões acerca de Métodos de Regularização; Planejamento e escrita do TCC I. \\

 \hline
 Junho & 
    Encontros para estudos e discussões acerca de Métodos de Regularização; Escrita e entrega final do TCC I.\\
    
 \hline
 Julho & 
    Apresentação do TCC I \\

\hline
 Agosto & 
    Escrita da seção sobre Tomografia por Impedância Elétrica; Escrita da seção de Métodos de Regularização e de Tomografia por Impedância Elétrica; Realização de Experimentos Numéricos \\

\hline
 Setembro & 
    Escrita das seções de Métodos de Regularização e Experimentos Numéricos; Realização de Experimentos Numéricos \\

\hline
 Outubro & 
    Escrita da seção de Teoria de Regularização e de Experimentos Numéricos; Realização e análise de Experimentos Numéricos \\

\hline
 Novembro & 
    Escrita dos tópicos de Introdução, Considerações Finais e Resumo; Revisões finais; Entrega da versão final \\

\hline
 Dezembro & 
    Apresentação de TCC II \\    
    
\hline
\end{tabular}
\end{center}
\end{table}

O cronograma descrito é passível de ajustes, à medida que se for percebendo a necessidade com o andamento do trabalho. Todavia, espera-se ao final do período estipulado o cumprimento de todos os procedimentos e etapas listadas.


% \begin{center}
% \begin{tabular}{c|ccccc}
% Tópico \textbackslash Mês & Agosto & Setembro & Outubro & Novembro & Dezembro \\
% \hline
%  Teoria\\ de Regularização  & & & & & \\ \hline
%  Métodos\\ de Regularização & & & & &  \\ \hline
%  Tomografia por\\ Impedância Elétrica & & & & & \\ \hline
%  Experimentos\\ Numéricos & & & & & \\ \hline
% \end{tabular}
% \end{center}
